\chapter{Ricerca degli iperparametri}

\paragraph{Introduzione}
In questa sezione della relazione verrà illustrato il problema principale su cui si è basato il tirocinio:
\textit{l'ottimizzazione della funzione di ranking di BM25P}.
Come illustrato nel capitolo 2, quasi ogni funzione di ranking possiede degli iperparametri, i quali
devono essere calibrati a seconda del tipo di dati e dalla natura dei dati stessi.
Per esempio se si dovesse calibrare BM25 per una collezione
di documenti scientifici medici, esso potrebbe avere degli iperparametri
differenti da quelli adatti per una collezione di documenti esclusivamente matematici.
Questo aspetto rende molto simile il mondo dell'Information Retrieval a quello del Machine Learning,
infatti alcuni termini e concetti sono di comune utilizzo.


Ricapitolando, gli iperparametri di BM25P sono:
\begin{table}[h!]
	\centering
	\begin{tabular}{|c|c|c|}
		\hline
		Parametro & Vincoli & Descrizione \\
		\hline
		$p$ & $p>1$ & numero di passages \\
		\hline
		$\alpha$ & $\alpha > 0$ & costante di amplificazione del term frequency \\
		\hline
		$\boldsymbol{w}$ & $\forall{i.1 \leq i \leq p}\implies \boldsymbol{w}_i \geq 0$ & vettore dei pesi dei passaggi \\
		\hline
	\end{tabular}
\caption{Iperparametri di BM25P}
\end{table}

Per questioni pratiche, la scelta che si è deciso di prendere è quella di cercare
nello spazio dei valori di $\boldsymbol{w}$, poiché esso rappresenta l'iperparametro
che agisce sui passages, cioè la particolarità introdotta da BM25P stesso, e lasciare
$p$ e $\alpha$ costanti.\footnote{Si potrebbe anche far vedere come variano i risultati usando un $p<10$, questo è da inserire nei lavori futuri}

\paragraph{La funzione obiettivo}

Come in qualsiasi ambito di ricerca in spazio di stati, anche in questo contesto
si deve scegliere una funzione da massimizzare, che però come vedremo presenterà dei problemi ben
noti, che limiteranno la scelta dei possibili algoritmi applicabili.
Una delle misure che meglio esprimono l'effettività di un motore di ricerca è NDCG \ref{def:ndcg}, la quale è
stata illustrata e commentata nel capitolo 3.
Massimizzando questa funzione siamo dunque in grado di ottenere una funzione di ranking che
è ottima per un certo dataset $\mathcal{D}$.

\paragraph{Obiettivo}
L'obiettivo che dunque bisogna porsi è quello di rispondere alle seguenti domande, ipotizzando
di avere $\mathcal{D}$:

\begin{enumerate}
	\item Quali sono gli iperparametri da impostare all'algoritmo per massimizzarne la qualità?
	\item Quali sono gli algoritmi per il calcolo di tali iperparametri?
	\item Di quanto possiamo aumentare l'efficienza di BM25 utilizzando l'estensione dei passages?
	\item Qual è la natura del vettore $\boldsymbol{w}$?
\end{enumerate}

Nei seguenti paragrafi verranno illustrati i meccanismi che sono stati
trovati, implementati e testati e ne verranno forniti anche alcuni risultati
sperimentali, alla fine della realazione inoltre è presente anche alcune
parti del codice Java.

\section{Setup sperimentale}

Una volta concluso lo studio di come Terrier è stato costruito e di come la sua pipeline è stata implementata,
il tirocinio si è rivolto nella ricerca di algoritmi adatti a fare \textit{model selection}.

\paragraph{Dataset}
Il dataset fornito insieme a Terrier, chiamato \textit{Vaswani}, è un dataset abbastanza piccolo,
contenente soltanto 10.000 titoli di documenti. Per poter avere dei risultati più
reali è stato utilizzato \textit{Aquaint}, acquistato con licenza dal C.N.R.
Esso contiene circa un milione di documenti e i relativi file di test, quali
un \textit{topic file} di 50 query e il \textit{qrels file} associato.
\textit{Aquaint} è un dataset prodotto da David Graff nel 2002,
contenente un elenco di news provenienti da tre fonti principali:
lo Xinhua News Service, il New York Times e l'Associated Press WorldStream News Service.


\paragraph{Macchine a disposizione}
Per poter accedere a tali dati, di dimensioni notevoli, è stato necessario utilizzare
delle macchine specifiche fornite dall'ente stesso, poiché sia per ragioni di sicurezza
che per ragioni pratiche, i dati non potevano essere resi pubblici.
Mi è stata dunque fornito dunque l'accesso ad una macchina 
all'interno della rete del C.N.R., in grado di sostenere il carico
di lavoro che gli algoritmi di ricerca avrebbero potuto richiedere.

\section{La pipeline di ricerca}
Per poter eseguire la massimizzazione della funzione obiettivo,
è necessario eseguire una serie di operazioni, che configurano il sistema Terrier
in modo da poter processare i dati con i parametri impostati.
Diamo di seguente una lista ordinata delle operazioni da eseguire:

\begin{enumerate}
	\item Inizializzare gli iperparametri scelti ad un valore iniziale
	\item Eseguire la fase di retrieve, producendo dunque una serie di risultati
	\item Eseguire la fase di evaluation, calcolando dunque il valore della funzione obiettivo
	\item Confrontare il valore attuale della funzione obiettivo con quello in memoria
	\item Aggiornare gli iperparametri e tornare al punto 2.
\end{enumerate}

Nel caso in questione l'iperparametro da variare è il vettore $\boldsymbol{w}$, la
quale lunghezza è uguale al numero dei passaggi, che abbiamo posto uguale a 10.
Pertanto gli algoritmi seguenti dovranno cercare all'interno di uno spazio di dimensione
10 il valore di $\boldsymbol{w}$ per cui NDCG è massima.

\paragraph{Alcune definizioni}
Il vettore dei pesi di BM25P è tale per cui ogni componente è un valore
positivo o nullo in $\mathbb{R}$. La ricerca dunque deve concentrarsi
in un intervallo chiuso che chiameremo $W_i$, dove l'indice $i$
indica l'indice della componente di $\boldsymbol{w}$.
Definiamo inoltre $w_{step}$ l'incremento da dare ad una o più
componenti ad ogni iterazione.
C'è anche la necessità di limitare inferiormente e superiormente
il valore che ogni componente di $\boldsymbol{w}$ può assumere,
poiché altrimenti lo spazio sarebbe infinito. Definiamo quindi
due vettori, $w_{start}$ e $w_{end}$ i quali rappresentano
i valore minimo e il valore massimo che $\boldsymbol{w}$
può assumere.

\section{Algoritmi completi}
Come primo tentantivo è stato pensato di usare una tipologia di algoritmi completi,
i quali provano tutti i valori possibili attribuibili e restituiscono quello ottimo. Un algoritmo
conosciuto per la sua semplicità è Grid Search, ovvero una ricerca esaustiva
su una griglia di valori, in questo caso una griglia $\Omega = W_1 \times W_2 \times \cdots \times W_{10}$.

\subsection{Grid Search}

Riporto per completezza lo pseudocodice di Grid Search:

\begin{algorithm}[h]
	\SetAlgorithmName{Algoritmo}{}
	\small
	\DontPrintSemicolon
	\SetKwInOut{Input}{Input}
	\SetKwInOut{Output}{Output}
	\Input{$w_{start} $ vettore minimo, di partenza}
	\Input{$w_{end}$ vettore massimo, di arrivo}
	\Input{$w_{step}$ incremento}
	\Output{$w_{opt}$ vettore dei pesi ottimo}
	\BlankLine
	$w_{opt} = w_{start}$\;
	$max_{eval} = -\infty$\;
	$w = w_{start}$\;
	\BlankLine
	\While{$w  \le w_{end}$}{
		terrierSetup($w$)\;
		terrierRetrieve()\;
		$ev$ = terrierEval()\;
		
		\If{$ev \ge max_{eval}$}{
			$max_{eval} = ev$\;
			$w_{opt} = w$\;
		}
		
		$w = w + w_{step}$\;
	}
	\Return{$\mathcal{I}$}
	\caption{\textsc{}}
	\label{alg:gss}
\end{algorithm}

Una precisione da tenere in considerazione è che il confronto
eseguito alla linea 4 è un confronto vettoriale, vale a dire che
viene confrontata ogni componente di $w$ con $w_{end}$ e se
è falso almeno un confronto, allora è falso anche il risultato.
L'incremento che viene fatto alla linea 11, inoltre, è un incremento
combinatorio. Ciò significa che si parte nel variare
l'ultima componente, quando si è raggiunto il massimo si
riporta al valore iniziale l'ultima e si incrementa la penultima,
e si esegue questo procedimento fino ad arrivare alla prima.

\begin{esempio}
Un esempio di incremento combinatorio, abbastanza banale, è il seguente.
Supponiamo di avere $w_{start} = \boldsymbol{0}$, $w_{end} = \boldsymbol{1}$
e $w_{step} = 0.5$ e $\left|w\right| = 4$.


\begin{table}[h]
	\centering
	\begin{tabular}{|c|c|}
		\hline
		Iterazione 1 & $w = \left[0, 0, 0, 0\right]$ \\
		\hline
		Iterazione 2 & $w = \left[0, 0, 0, 0.5\right]$ \\
		\hline
		Iterazione 3 & $w = \left[0, 0, 0, 1\right]$ \\
		\hline
		Iterazione 4 & $w = \left[0, 0, 0.5, 0\right]$ \\
		\hline
		$\cdots$ & $\cdots$ \\
		\hline
		Iterazione 16 & $w = \left[1,1,1,1\right]$ \\
		\hline
	\end{tabular}
	\caption{Esempio incremento combinatorio}
\end{table}
\end{esempio}

Utilizzando questo tipo di incremento è possibile scorrere
tutto lo spazio dei valori, dunque la ricerca è completa e anche ottima
su $\Omega$.
Il problema principale di questo tipo di algoritmo è che
per poter rendere l'algoritmo eseguibile in un tempo finito
è necessario che $\Omega$ sia finito.

\begin{definizione}
	Le condizioni di completezza e ottimalità ideale per l'algoritmo Grid Search sono:
	\begin{enumerate}
		\item $w_{start} = \boldsymbol{-\infty}$ illimitato inferiormente
		\item $w_{end} = \boldsymbol{+\infty}$ illimitato superiormente
		\item $\lim\limits_{h \to 0} \left(w_{step} - h\right) = 0$ continuo
	\end{enumerate}
\end{definizione}

Com'è possibile notare un algoritmo che soddisfi tali condizioni è
ineseguibile, sia per questioni di tempo che per questioni di spazio.
Ciò che però è possibile fare è applicare una restrizione ad $\Omega$, attraverso
una limitazione degli estremi e una discretizzazione (il cui ruolo è giocato da $w_{step}$).

\subsection{Analisi spazio}

Prima di imbatterci in una ricerca che non terminerebbe nei tempi previsti è necessario
eseguire un'analisi dettagliata dello spazio di ricerca, considerando anche il budget temporale.
Chiamiamo con $\mathcal{B}$ tale budget a disposizione, che esso
può variare in base alle esigenze dell''amministratore
del motore di ricerca.

Si ha anche che il tempo di esecuzione della Pipeline (Retrieve, Eval) è quantificato
da $\tau$, che possiamo considerarlo costante una volta fissati la collezione dei dati e gli insiemi di test.

A questo punto diremo che la dimensione dello spazio da analizzare
è dato da:

\begin{align*}\label{eq:gsdim}
& \Omega = \bigtimes_{i=1}^{p} W_i \\
& \text{Dim}(\Omega) = \prod\limits_{i=1}^{p} \frac{w_{end_i} - w_{start_i}}{w_{step}} \\
\end{align*}

\'E immediato specializzare tale equazione nel caso in cui i vettori $w_{start}$ e $w_{end}$
siano esprimibili come $w_{start|end} = \vec{r}$ dove $r \in \mathbb{R}$. \footnote{
	Si intendono tutti i vettori le quali componenti sono identiche le une alle altre, per esempi o il vettore $v = \left[0, 0, 0\right]$.
	Una variante potrebbe essere quella di scrivere che $\forall{i,j.1 \leq i,j \leq p} \implies x_i = x_j$.
}


Si ottiene che:

$$
\text{Dim}(\Omega) = \left(\frac{w_{end} - w_{start}}{w_{step}}\right)^{p}
$$\footnote{Con Dim di solito si intende la dimensione dello spazio quindi sarebbe $p$, in questo caso però intendo la grandezza, cioè quanti elementi ci sono, forse è meglio definirlo come $\#\Omega$?}

Dunque lo spazio risulta esponenzialmente grande quanto lo sono il numero
dei passaggi $p$, e questo è uno dei problemi che portano Grid Search ad essere
un algoritmo troppo pesante per poter essere eseguito, poichè:

$$
\mathcal{B} \geq \tau \cdot \text{Dim}(\Omega)
$$

Siccome $\tau > 0$, questo meccanismo di ricerca risulta poco efficiente, e quindi non è consigliato.

\begin{esempio}\label{ex:gsspace}
	Sia $\tau = 5s$ il tempo che Terrier impiega per eseguire tutta la pipeline, e sia $p=10, w_{start} = \vec{0}, w_{end} = \vec{3}, w_{step} = 5 \cdot 10^{-3}$.
	Allora si ha che
	$$
	\mathcal{B} \geq \left(\frac{3}{5 \cdot 10^{-3}}\right)^{10} \approx 6 \cdot 10^{27}s \approx 10^{20} \text{anni}
	$$
\end{esempio}


\begin{esempio}
	Si può anche tenere di conto della formula inversa, cioè avere a disposizione $\mathcal{B}$ e voler
	calcolare $w_{step}$. Per scopi puramente sperimentali possiamo eseguire questo calcolo.
	
	Riprendiamo i soliti dati dell'esempio precedente, e teniamo come incognita $w_{step}$, ipotizzando
	invece $\mathcal{B} = 24$ ore.
	Si ha che:
	$$
	24 \cdot 3600 = \left(\frac{3}{w_{step}}\right)^{10}
	$$
	e quindi
	$$
	w_{step} \approx 0.9
	$$
	
	Ma in questo modo la discretizzazione dello spazio sarebbe troppo evidente e dunque la ricerca è poco
	probabile che trovi l'ottimo.
\end{esempio}

\section{Algoritmi informati}

\subsection{Euristica del gradiente}
Per poter massimizzare NDCG, si potrebbe pensare di utilizzare il metodo di salita del gradiente,
ma quest'ultimo risulta di difficile implementazione, poiché la funzione vettoriale
da derivare risulta essere ardua. Non è stato escluso però che utilizzando tecniche
sofisticate si possa essere in grado di tener conto anche dell'informazione
che il gradiente potrebbe dare.
Tale argomento potrebbe essere sviluppato in una ricerca futura.

\subsection{Increase Search}
L'ultimo algoritmo informato che ho deciso di esporre è un algoritmo che ho
progettato personalmente, tenendo in considerazione sia del metodo di salita rapida,
sia del metodo di salita del gradiente,
sia della ricerca esaustiva. Esso consiste in un meccanismo di calcolo del trend della funzione obiettivo,
ovvero di un surrogato della derivata di NDCG, che dà una minima informazione su come
muoversi nello spazio $\Omega$.

L'algoritmo verrà illustrato per passi nei seguenti paragrafi.

\paragraph{Calcolo del trend}
Per calcolare il trend è sufficiente raccogliere una serie di dati formati
da coppie $\langle w, e\rangle$, dove $e$ è il valore di NDCG calcolato
utilizzando $w$. Successivamente l'idea è quella di capire
l'andamento della successione dei valori di $e$ mediante
la sottrazione dell'elemento $i$-esimo con l'elemento ($i-1$)-esimo.
Si calcola dunque il vettore delle \textit{differenze} $dE$, di lunghezza $p-1$, nel seguente modo:
$$
\forall{i.2 \leq i \leq p} \quad dE_{i-1} = e_{i} - e_{i-1}
$$

Attraverso $dE$ è possibile dare un'informazione del tutto
\textit{osservativa} dell'andamento di $e$, per esempio
con una somma discreta:

$$
\text{trend}(e) = \sum_{i=1}^{p-1} \text{sign}(dE_i)
$$

Per avere a disposizione un informazione del tipo crescita/decrescita/instabilità,
si può considerare la funzione

$$
\text{trend}(e) = \text{sign}\left(\sum_{i=1}^{p-1} \text{sign}(dE_i)\right)
$$

ed in questo modo:
$$
\text{trend}(e) =
\begin{cases}
-1 & \text{se la funzione è definitivamente decrescente} \\
0 & \text{se la funzione è instabile} \\
+1 & \text{se la funzione è definitivamente crescente} \\
\end{cases}
$$

\subparagraph{Trend pesato}
Una versione più sofisticata del trend (ma che non è stata utilizzata)
è la versione pesata, che favorisce le salite ``più vicine"".

$$
\text{trend}_p(e) = \text{sign}\left(\sum_{i=1}^{p-1} \text{sign}\left(dE_i \cdot (p-i-1)\right)\right)
$$


\paragraph{Ricerca}
La ricerca di Increase Search procede scegliendo in ordine da
un insieme di indici $\mathcal{I}$ una componente su cui eseguire
l'incremento, fino a che non si decide o di abbandonare
l'opportunità oppure di proseguire, scegliendo quindi la componente successiva.
Sia dunque $\mathcal{P}$ l'insieme delle permutazioni della stringa
composta da tutti gli indici di $w$, quindi:

$$
\mathcal{P} = \text{Perm}(\langle1,\cdots, p\rangle)
$$

Scegliendo un elemento $p \in \mathcal{P}$, l'algoritmo
inizia incrementando la componente $p_1, p_2, \cdots$ una alla volta, fino a che
non è in grado di ``prendere una decisione".
Una volta completata l'iterazione sull'insieme $p$, si passa all'insieme all'insieme successivo.

\paragraph{Valutazione}
Nel paragrafo precedente si è parlato imprecisamente di
``prendere una decisione", ma cosa si intende, e come si può
implementare?
Prendere una decisione alla fine dell'iterazione significa essenzialmente
calcolare il trend e decidere se proseguire con l'incremento oppure
fermarsi. L'algoritmo implementato prevede un meccanismo di bonus,
vale a dire un meccanismo che consente alla funzione di avere un andamento
decrescente ma dopo svariati incrementi iniziare a crescere.
Viene dunque richiesto un parametro $b_{max} \geq 0$, il quale rappresenta
il numero massimo di bonus, il quale viene confrontanto continuamente con il
numero di bonus attuali.

\paragraph{Algoritmo completo}
Di seguente illustro lo pseudocodice dell'algoritmo completo:

\begin{algorithm}[h!]
	\SetAlgorithmName{Algoritmo}{}
	\small
	\DontPrintSemicolon
	\SetKwInOut{Input}{Input}
	\SetKwInOut{Output}{Output}
	\Input{$w_{start} $ vettore minimo, di partenza}
	\Input{$w_{end}$ vettore massimo, di arrivo}
	\Input{$w_{step}$ incremento}
	\Input{$p \in \mathcal{B}$ permutazione degli indici}
	\Input{$et_{min}$ numero di elementi minimi per calcolare il trend}
	\Output{$w_{opt}$ vettore dei pesi ottimo}
	\BlankLine
	$w_{opt} = w_{start}$\;
	$max_{eval} = -\infty$\;
	$w = w_{start}$\;
	
	\BlankLine
	\ForEach{$i \in p$}{	
		$b = 0$\;
		\While{$w_i \le w_{end_i}$} {
		terrierSetup($w$)\;
		terrierRetrieve()\;
		$ev$ = terrierEval()\;
		\BlankLine
		\If{$ev \ge max_{eval}$}{
			$max_{eval} = ev$\;
			$w_{opt} = w$\;
		}
		updateTrend($ev$)\;
		\BlankLine		
		\If{trendReady()}{
			$t = $ getTrend()\;
			
			\If{$t = 0 \vee t = -1$}{
				$b = b +1$\;
			}
			
			\If{$b \ge b_{max}$}{
				rollBack()\;
				break\;
		}
		}
		\BlankLine
		$w_i = w_i + w_{step}$\;
		}
	}
	
	\Return{$\mathcal{I}$}
	\caption{\textsc{}}
	\label{alg:gs}
\end{algorithm}

La funzione \textit{isTrendReady()} è un funzione booleana
che consente di far sapere all'algoritmo se i dati che sono stati memorizzati
sono sufficienti a produrre un trend oppure no. Ciò è regolato da un
parametro impostato prima della computazione, il cui valore
suggerito, in accordo con la grandezza dello spazio, è circa 10.

Anche la funzione \textit{getTrend()} gioca un ruolo fondamentale,
perché oltre a calcolare il trend cancella la lista delle coppie
$\langle w, e\rangle$.

Da notare che è presente anche un meccansimo di backtracking,
alla linea \textit{19}, il quale non fa altro che ripristinare
i valori per la quale la funzione obiettivo era massima.

\paragraph{Problemi di Increase Search}
Il problema principale di questo algoritmo è che esso si limita a ricercare
in uno spazio discreto, come Grid Search, e la sua complessità è dell'ordine
del fattoriale, poiché esso per avvicinarsi all'ottimalità dovrebbe
provare tutte le permutazioni degli indici, che sono esattamente $p!$.
Inoltre l'euristica del trend è limitata, in quanto si limita
ad osservare un andamento del tutto empirico della funzione obiettivo.

\section{Algoritmi randomizzati}
Un modo alternativo per fare model selection è quello di utilizzare degli algoritmi randomizzati, cioè
degli algoritmi che hanno un meccanismo di randomizzazione che consente di evitare
il problema dei massimi locali, poiché vengono utilizzati algoritmi di salita rapida.

\subsection{Line search with Random Restart}
L'algoritmo di Line Search è un algoritmo di hill-climbing che consiste nell'eseguire
una ricerca su griglia, ma con un random restart per ogni iterazione.
L'idea alla base è la seguente:

\begin{enumerate}
	\item Si inizializza $w$ con dei valori casuali $c_i \in W_i$.
	\item Si sceglie una componente e si inizia ad incrementare, fino ad arrivare al massimo
	\item Si sceglie la componente successiva e si esegue il punto 2.
	\item Quando le componenti sono state iterate tutte, si ritorna al punto 1.
\end{enumerate}

Tale operazioni possono essere eseguite fino a che o non si è trovato un valore
soddisfacente della funzione obiettivo o fino all'esaurirsi del tempo a disposizione.

\pagebreak

Lo pseudocodice dell'algoritmo è il seguente:
\begin{algorithm}[h!]
	\SetAlgorithmName{Algoritmo}{}
	\small
	\DontPrintSemicolon
	\SetKwInOut{Input}{Input}
	\SetKwInOut{Output}{Output}
	\Input{$w_{start} $ vettore minimo, di partenza}
	\Input{$w_{end}$ vettore massimo, di arrivo}
	\Input{$w_{step}$ incremento}
	\Output{$w_{opt}$ vettore dei pesi ottimo}
	\BlankLine

	$w_{opt} = w_{start}$\;
	$max_{eval} = -\infty$\;
	$w$ = randomW()\;
		
	\BlankLine
	
	\For{$i =1 $ to $i=p$}{
		\While{$w_i \le w_{end_i}$}{
		terrierSetup($w$)\;
		terrierRetrieve()\;
		$ev$ = terrierEval()\;
		\BlankLine
		\If{$ev \ge max_{eval}$}{
			$max_{eval} = ev$\;
			$w_{opt} = w$\;
		}
		$w_i = w_i + w_{step}$\;
	}
	}
	
	\Return{$w_{opt}, max_{eval}$}
	\caption{\textsc{}}
	\label{alg:lsrs}
\end{algorithm}



\begin{algorithm}[h!]
	\SetAlgorithmName{Algoritmo}{}
	\small
	\DontPrintSemicolon
	\SetKwInOut{Input}{Input}
	\SetKwInOut{Output}{Output}
	\Input{$\mathcal{B} $ budget temporale}
	\Input{$\mathcal{M} $ massimo numero di iterazioni}
	\Output{$w_{opt}$ vettore dei pesi ottimo}
	\BlankLine
	
	$max_{eval} = -\infty$\;
	$w_{opt} = \emptyset $\;
	\While{budgetAvailable($\mathcal{B}$) $\wedge \quad i<\mathcal{M}$}{
				$w, ev$ = LineSearchRandomRestart($\dots$)\;
					\If{$ev \ge max_{eval}$}{
					$max_{eval} = ev$\;
					$w_{opt} = w$\;
		}
	}
	
	\Return{$w_{opt}$}
	\caption{\textsc{}}
	\label{alg:lsrsm}
\end{algorithm}

Da notare che la chiamata dell'algoritmo 4, è gestita
dall'algoritmo 5. L'algoritmo 5 tiene conto del budget temporale e mantiene in memoria
il massimo valore ritornato, in modo da restituire alla fine dell'esecuzione
il valore dei pesi che rende massima la funzione obiettivo.

\paragraph{Svantaggi}
Gli svantaggi utilizzando questi metodi sono essenzialmente tutti dovuti al budget temporale,
poiché per essere sicuri di trovare la soluzione ottima $\mathcal{B} \to +\infty$.


