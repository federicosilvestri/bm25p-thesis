\chapter{Le funzioni di scoring}
In questo capitolo verranno illustrate le due funzioni di scoring principali su cui si è basato il tirocinio.
Verrà descritta in primo luogo BM25 e successivamente la sua estensione BM25P, la quale sarà l'oggetto
dello studio.

\section{BM25}
Okapi BM25 è una funzione di scoring basata sul framework probabilistico sviluppato tra il 1970 e il 1980
da Stephen E. Robertson e Karen Jones. La sua prima applicazione fu nel sistema di ricerca Okapi, il quale
fu implementato dalla London's City University. La funzione soddisfa i requisiti illustrati nel capitolo 1, ed è definita
nel seguente modo:

\begin{definizione}\label{def:bm25}
	$$
	\sco_{BM25}(d, q) = \sum_{i=1}^{|q|} \text{idf}(q_i, \mathcal{D}) \cdot r_i
	$$
	
	e
	$$
	r_i = \frac{
		\tf(q_i, d) \cdot (k_1 +1)
	}{
		\tf(q_i, d) + k_1 \cdot \left(1-k_2 + k_2 \cdot \frac{\left|d\right|}{\text{avgdl}}\right)
	}
	$$
	
	
	dove abbiamo che:
	
	\begin{table}[h]
	\begin{tabular}{|l|p{110mm}|}
		\hline
		$k_1, k_2$ &  sono iperparametri, \newline tipicamente $k_1 \in\left[1.2, 2.0\right]$ e $k_2 \approx 0.75$  \\
		\hline
		$\text{avgdl}$ &  è la media aritmetica della lunghezza dei documenti nella collezione $\mathcal{D}$  \\
		\hline
	\end{tabular}
\end{table}
\end{definizione}


L'iperparametro $k_1$ regola il contributo del term frequency, ovvero impostando $k_1 = 0$ avremo
che il modello sarà di tipo binario, poiché avremo $\forall i . 1 < i < \left|q\right| \implies r_i = 1$.
Mentre impostandolo ad un valore elevato avremo esattamente un modello di tipo raw count, poiché
si terrà molto in considerazione delle singole frequenze.
Il parametro $k_2$ invece regola la normalizzazione della lunghezza del documento, più è grande,
più il documento tende ad esser normalizzato.
Da notare che il parametro $\mathcal{D}$ è implicito nella funzione, infatti si dà per scontato che
la funzione di scoring venga
eseguita una volta effettuato l'indexing, in modo da avere noto anche $\text{avgdl}$.
Per comprendere al meglio tale funzione, si consideri la formula come una combinazione
lineare dei rapporti $r_i$. Espandendo la sommatoria, si ha quindi
$$
\idf(q_1, \mathcal{D}) \cdot r_1 + \dots + \idf(q_n, \mathcal{D}) \cdot r_n
$$
La caratteristica di questi modelli infatti è proprio il fatto di esprimere lo score come una somma pesata
di fattori, nel caso di BM25 dei vari $r_i$,  cioè del rapporto tra le frequenze. \footnote{Qui forse dovrei spiegare che i rapporti sono i pesi $w_i$, ma non volevo fare confusione con il vettore dei pesi $\vec{w}$ di BM25P}

\subsection{Limitazioni di BM25}

Il problema di questa funzione è che essa non considera la struttura dei documenti, cioè ipotizza
la non strutturazione.
Ma a seconda del tipo di collezione, i documenti possono essere strutturati, infatti
se consideriamo una collezione di articoli scientifici avremo che un singolo documento è composto da un titolo, 
un'introduzione, alcuni paragrafi e una conclusione.
Oppure possiamo considerare le pagine del web: tutte le pagine sono composte da un titolo e da un corpo.
Per questo motivo è stato pensato di aggiungere una meccanismo per suddividere un documento in
più parti ed associare ad ogni parte un peso specifico.

Questa è stata l'idea di BM25F, cioè BM25 con l'aggiunta dei "fields", ovvero dei campi che caratterizzano ogni documento in collezione.
Sotto l'assunzione forte che i documenti abbiamo tutti la solita struttura, quindi i soliti campi,
possiamo attribuire ad ogni campo un peso  $w_i$.
Per esempio $w_{titolo} = 2.5, w_{corpo} = 1.0$, in modo da aumentare il contributo del titolo.

\section{BM25P}
Il problema risolto da BM25F ha come ipotesi il fatto che tutti i documenti siano strutturati con una struttura comune.
In realtà non sempre si ha a che fare che questi tipi di documenti e la memorizzazione dei campi comporterebbe, durante
la fase di indexing, una complessità addizionale sia in spazio che in tempo.
Possiamo però tenere di conto che se ogni documento è rappresentato come \textit{raw text}, ovvero come una sequenza
di token, allora è possibile inferire la sua struttura in base alla posizione dei token.
Questo significa che se la posizione del token $t$ è vicina allo zero allora esso sarà contenuto nel titolo, così come
se $t \approx \left|d\right|$, allora $t$ farà parte dell'ultimo paragrafo del documento.
L'idea alla base è quella dunque di suddividere un documento in $p$ \textit{passages}, con $p>1$, e
considerare ogni passage come un campo. A questo punto possiamo riprendere l'idea di BM25F
ed estendere la formula utilizzando al posto dei campi i passages.

\begin{definizione}[BM25 Passages]\label{def:bm25p}
		$$
	\sco_{\scaleto{BM25P}{4pt}}(d, q) = \sum_{i=1}^{|q|} \text{idf}(q_i, \mathcal{D}) \cdot r_i
	$$
	
	e
	
	$$
	r_i = \frac{
		\tfp(q_i, d) \cdot (k_1 +1)
	}{
		\tfp(q_i, d) + k_1 \cdot \left(1-k_2 + k_2 \cdot \frac{\left|d\right|}{\text{avgdl}}\right)
	}
	$$
	
	e
	
	$$
	\tfp = \sum_{i \in \mathcal{P}(d))}{\alpha \boldsymbol{w}_i \cdot \tf(i, d)}
	$$
	
	Dove $\alpha$ è un iperparametro precomputato per scalare la frequenza del termine, solitamente impostato $\alpha = 10$.
	Da notare anche l'introduzione del vettore $\boldsymbol{w}$, che gioca il ruolo di amplificare il valore delle frequenze
	in base al passage in questione.
	Tale vettore ha tante componenti quanti sono i passaggi, e possiamo anche stabilire una proporzionalità
	diretta tra $\boldsymbol{w}_i$ e la distrubuzione di probabilità del termine $t$ nel passage $i$.
	
	Si noti infine che impostando $\alpha=1$ e $\boldsymbol{w} = \boldsymbol{1}$, BM25P è uguale a BM25.
\end{definizione}

\section{I weighting models a confronto}
 
 Abbiamo visto abbastanza in dettaglio entrambe le funzioni di scoring e in questo paragrafo possiamo metterle a confronto.
 BM25 è un meccanismo che tiene in considerazione documenti non strutturati (possiamo anche chiamarli fluidi),
 mentre BM25F tiene in considerazione di documenti strutturati tutti nel solito modo.
 Sfruttando la troppa vincolatività di BM25F e la troppa flessibilità di BM25, l'idea di BM25P è quella che meglio
 si adatta per documenti che sono fluidi, ma hanno nella loro natura una suddivisione implicita.
Questo perché la funzione $\mathcal{P}(d)$ di  BM25P suddivide in modo semiautomatico
 i documenti in passaggi, inferendo dunque una struttura a \textit{sezioni} o passages.
 Se ipotizzassimo di porre $p=3$,
 avremmo che ogni documento in collezione verrà suddiviso in 3 parti, che possono corrispondere
 a introduzione, paragrafo centrale e conclusioni. Naturalmente questa approssimazione deve essere
 gestita in base alla lunghezza media dei documenti, perché più un documento è grande più passaggi possono
 essere richieste per avere una buona efficienza di BM25P. Un modo per calcolare il numero di passaggi è
 stato inoltre illustrato nelle prime sezioni dell'articolo di BM25P\cite{10.1145/3331184.3331373}.
 