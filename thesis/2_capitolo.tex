\section{Architettura generale}
In questo capitolo verrà illustrata l'architettura generale di un sistema di IR e verranno
descritti alcuni processi che sono stati oggetto di studio del tirocinio.
Quando si parla di sistema di IR, oppure in termini più quotidiani di \textit{search engine},
siamo spesso abituati a vederlo come un sistema di interrogazione dove l'utente
digita come input un testo, chiamato query e il motore risponde con una serie di risultati. Tali risultati
sono visualizzati all'utente come una lista di documenti che contengono informazioni
rilevanti secondo la query.
\`E immediato pensare all'esempio di Google, dove l'utente digita una serie di termini
all'interno del  form e dopo qualche secondo gli vengono restituiti tutti i link a pagine web che
sono in qualche modo collegate a ciò che l'utente stava realmente cercando.
Quello che sta dietro a tutto questo processo, che sembra essere quasi immediato nell'esempio di Google, è
in realtà un lavoro molto complesso e lungo.
La fase che l'utente è abituato ad esserne partecipe è la fase finale, ovvero
il \textit{Retrieving}. Ma prima di poter parlare di ciò è necessario illustrare la fase primaria, l'\textit{Indexing}.

\paragraph{Fase di indexing}
Per poter offrire quella velocità che si richiede durante la fase di retrieving è necessario aver "preparato"
una struttura dati che sia completa, ovvero che comprenda tutte le informazioni necessarie e 
rapida da scorrere, ovvero che le informazioni che si stanno cercando devono poter essere recuperate
il prima possibile.  Un esempio abbastanza esplicativo è quello dell'addetto alla biblioteca che deve rispondere
alle richieste degli studenti. Se l'addetto non conoscesse i libri che sono parte della biblioteca per ogni richiesta
dello studente dovrebbe scorrere tutta la lista e capire se quel dato libro può essere utile oppure no.
Questo esempio rappresenta il fatto che l'addetto non è in possesso di una struttura in grado
di accedere solo alle informazioni necessarie per la ricerca.

La costruzione di tale struttura dati, chiamata indice, è la fase più dispendiosa in termini di memoria e tempo di elaborazione.
Questo perché si devono analizzare tutti i documenti
che fanno parte della collezione ed estrapolarne le informazioni da inserire nell'indexer.
Si noti che una modifica alla collezione, come potrebbe essere l'aggiunta di un documento o la rimozione, comporta
la reindicizzazione.
Durante la fase di Indexing tutti i documenti che sono contenuti all'interno della collezione vengono analizzati e ne viene costruito l'indice.
Mentre durante la fase di Retrieving si ha la cosìdetta interrogazione del motore di ricerca da parte degli utenti.



\begin{algorithm}[h]
	\small
	\DontPrintSemicolon
	\SetKwInOut{Input}{Input}
	\SetKwInOut{Output}{Output}
	\Input{$\mathcal{D} $ collezione di documenti}
	\Output{$\mathcal{I}$ indice}
	\BlankLine
	\ForEach{$d \in \mathcal{D}$}{
		$tokenizer = getTokenizer(d)$\;
		\While{$token = tokenizer.nextToken()$}{
			$updateIndex(token, d)$
		}
	}
	\Return{$\mathcal{I}$}
	\caption{\textsc{}}
	\label{alg:bray-curtis}
	
\end{algorithm}
