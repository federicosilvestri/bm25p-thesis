\chapter{Risultati sperimentali}

In questo capitolo verranno esposti i risultati che sono stati trovati
utilizzando il dataset \textit{Aquaint} e il relativo test set.

\section{Valori di riferimento}
Come prima cosa è necessario capire sul dataset in questione
quali sono le prestazioni ottenibili utilizzando l'algoritmo rivale di BM25P,
BM25. I seguenti risultati sono stati calcolati con gli iperparametri standard
di BM25.

\begin{table}[h!]
	\centering
	\begin{tabular}{|c|c|}
		\hline
		recall@100 & 0.2071 \\
		\hline
		recall@200 & 0.3069 \\
		\hline
		NDCG & 0.4161 \\
		\hline
		NDCG@5 & 0.2800 \\
		\hline
		NDCG@10 & 0.2707 \\
		\hline
	\end{tabular}
\caption{Valutazione di BM25}
\end{table}

Con la notazione $misura@k$ si intende che
la misura è stata effettuata sui primi $k$ risultati, in gergo tecnico
è chiamato taglio.

\section{Algoritmo Grid Search}
Utilizzando l'agoritmo Grid Search con i seguenti valori:

\begin{table}[h!]
	\centering
	\begin{tabular}{|c|c|}
		\hline
		$w_{start}$ & $[0.5, 0.5, 0.5, 0.5, 0.5, 0.5, 0.5, 0.5, 0.5, 0.5]$ \\
		\hline
		$w_{end}$ & $[5.0, 5.0, 5.0, 5.0, 5.0, 5.0, 5.0, 5.0, 5.0, 5.0]$ \\		
		\hline
		$w_{step}$ & 0.25 \\
		\hline
	\end{tabular}
\caption{Configurazione di GridSearch}
\end{table}

I valore massimi delle funzioni di valutazione trovati
sono i seguenti:

\begin{table}[h!]
	\centering
	\begin{tabular}{|c|c|}
		\hline 
		recall@100  &  0.1932  \\
		\hline
		recall@200   & 0.2786 \\
		\hline
		NDCG    & 0.3845 \\
		\hline
		NDCG@5  & 0.3246 \\
		\hline
		NDCG@10  & 0.2952 \\
		\hline
		\multicolumn{2}{|c|}{$w = [1.531, 0.843, 0.863, 0.861, 0.856 , 0.912, 0.874, 0.856, 0.865, 1.534]$}
	\end{tabular}
	\caption{Risultati di GridSearch}
\end{table}

Grid Search dopo svariate ore di lavoro non è riuscito a produrre risultati soddisfacenti.

\section{Line Search with Random Restart}

\begin{table}[h!]
	\centering
	\begin{tabular}{|c|c|}
		\hline
		recall@100 &  0.1988 \\
		\hline
		recall@200 & 0.2793  \\
		\hline
		NDCG & 0.3814 \\
		\hline
		NDCG@5 & 0.3188 \\
		\hline
		NDCG@10 & 0.2901 \\
		\hline
		\multicolumn{2}{|c|}{
			$w = [1.071, 0.821, 0.821, 0.821, 0.821, 0.821, 1.071, 0.821, 1.0714, 1.071]$
		}
	\end{tabular}
	\caption{Risultati di Line Search with Random Restart}
\end{table}

\section{Increase Search}

\begin{table}[h!]
	\centering
	\begin{tabular}{|c|c|}
		\hline
		recall@100 &  0.1971  \\
		\hline
		recall@200 & 0.2954  \\
		\hline
		NDCG & 0.4119 \\
		\hline
		NDCG@5 & 0.3542 \\
		\hline
		NDCG@10 & 0.3007 \\
		\hline
		\multicolumn{2}{|c|}{$w = [0.1981, 0.1012, 0.07532, 0.075321, 0.15023, 0.12532, 0.12512, 0.103, 0.0521, 0.2534]$}
	\end{tabular}
	\caption{Risultati di Increment Search}
\end{table}

\section{Test di significatività statistico}
Per ogni query del topic file, è stata eseguita la pipeline (retrieve->eval) per BM25 e BM25P. Le misure di valutazione per query scelte sono state NDCG, NDCG@5 e NDCG@10.


|Measure|$p_1$     | $p_2$    |
|:-----:|-------|-------|
|NDCG   |0.13218|0.26347|
|NDCG@5 |0.00082|0.00145|
|NDCG@10|0.03454|0.06835|


Considerando che solitamente viene scelto un $\alpha$ < 0.05, oppure per essere più restrittivi $\alpha$ < 0.01, la *null-hypothesis* (BM25P $\equiv$ BM25) può essere considerata falsa.
Questo perché, dato che l'ottimizzazione di BM25P è stata basata sui tagli più piccoli, il test di significatività statistico su NDCG@5 e NDCG@10 deve avere più peso rispetto a NDCG. Questo risultato inoltre ci mostra che i due modelli, su tagli più grandi, potrebbero essere simili, in quanto $p_2$ > $p_1$ > 0.05.
