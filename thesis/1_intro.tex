\chapter{Introduzione}
Al giorno d'oggi cercare informazioni sul web è diventata una delle operazioni più comuni e più richieste:
utilizzare lo smartphone per eseguire una ricerca sui punti di interesse di una città, o per scoprire la ricetta di un dolce è capitato quasi a tutte le persone.
Le statistiche elaborate direttamente da Google affermano che ogni secondo vengono richieste circa \textit{40.000} query, per un totale
giornaliero di \textit{3,5 miliardi}.
Siamo arrivati persino a parlare di memoria \textit{transattiva}, ovvero di una memoria condivisa tra tutti gli utenti e disponibile attraverso il web.
Infatti l'essere umano che possiede un qualunque mezzo tecnologico è in grado di accedervi in svariati modi, per esempio facendo una ricerca tramite Siri, Google oppure Alexa.
Tale memoria non consiste nel memorizzare l'informazione stessa, ma in come giungere ad essa.
Ma non è solo all'interno del web che c'è la necessità di un sistema di ricerca. Si pensi per esempio a Spotlight di Mac OS, dove
viene utilizzato  per file di testo, audio, immagini e anche news dalla rete.
Il ruolo dell'Information Retrieval in questi giorni è dunque di fondamentale importanza.
Per questo gli algoritmi si sono dovuti sviluppare in tutte le dimensioni: a partire dall'architettura delle macchine sulle quali i documenti vengono archiviati a come i documenti stessi vengono strutturati. In questa tesi di tirocinio verrà illustrato \textit{Terrier}, una piattaforma sviluppata dall'università di Glasgow
per sperimentare gli algoritmi legati all'Information Retrieval, anche su collezioni di dati nell'ordine dei terabyte.  Successivamente il focus si sposterà
sugli algoritmi di ranking, dei quali andremo a utilizzare BM25P, una versione di BM25 migliorata dal gruppo ISTI del CNR di Pisa.
Verranno illustrati una serie di algoritmi il cui scopo è quello di ricercare all'interno di uno spazio multidimensionale
gli \textit{iperparametri} di BM25P, incrementando dunque le prestazioni rispetto a BM25 stesso.
Attraverso dei test statistici verificheremo infine che i due modelli di ranking sono decisamente diversi ed è possibile
inferire in quali \textit{passages} sono contenute le informazioni più \textit{rilevanti}, in forte relazione al tipo di collezione di documenti.