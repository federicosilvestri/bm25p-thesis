Al giorno d'oggi ricercare informazioni per mezzo di un motore di ricerca è diventata una delle operazioni più comuni e più richieste.  Utilizzare lo smarphone per
eseguire una ricerca sui punti di interesse di una città, o per scoprire la ricetta di un dolce è capitato quasi a tutte le persone. Siamo arrivati persino a parlare di memoria \textit{transattiva}, ovvero di una memoria condivisa tra tutti gli utenti e disponibile attraverso il web. Infatti l'essere umano che possiede un qualunque mezzo tecnologico è in grado di accedervi in svariati modi, per esempio facendo una ricerca tramite Siri, Google oppure Alexa.
Tale memoria non consiste nel memorizzare l'informazione, ma come giungere ad essa tramite un motore di ricerca.
Per questo gli algoritmi di Information Retrieval si sono dovuti adattare in tutte le dimensioni: a partire dall'architettura delle macchine sulle quali i documenti vengono archiviati a come i documenti stessi vengono strutturati.
