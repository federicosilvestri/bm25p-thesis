\chapter{Descrizione del tirocinio}

Il tirocinio si è svolto quasi sempre in modalità telematica,
sia perché la situazione dovuta al COVID-19 non ha concesso
gli spostamenti, sia perché  il lavoro che doveva essere svolto
era per lo più individuale.
Sono stato supervisionato principalmente da tre persone,
le quali mi hanno dato degli obiettivi da raggiungere per ogni
step del tirocinio,
fino ad arrivare alla conclusione.
Il codice implementato risiede in piccola
quantità nell'appendice di questa relazione,
e invece tutto il progetto e tutte le sperimentazioni
sono disponibili sulla pagina personale di \href{https://github.com/federicosilvestri/bm25p-thesis}{GitHub}.

Il tirocinio si è concluso una volta raggiunti i risultati
previsti, e le ore totali impiegate sono state circa 320.

\subsection{Basi di partenza}
Questo tipo di tirocinio si è basato
sullo studio di una materia non trattata nel corso di laurea triennale:
\textit{Information Retrieval} (\textit{IR}).
Tale materia è risultata abbastanza facile da comprendere grazie
al materiale fornito dai tutor e alla loro disponibilità.
Una volta approfondito lo studio è iniziata la fase sperimentale,
cioè la progettazione e l'implementazione di svariati meccanismi per la ricerca 
degli iperparametri ottimi di BM25P.

L'algoritmo BM25P è un'estensione della funzione di ranking
di BM25, un noto algoritmo nel ramo dell'\textit{IR}.
BM25P è stato ideato e pubblicato su ACM dal C.N.R.
stesso nel 2019.\cite{10.1145/3331184.3331373}

Per verificare la validità di ciò che è stato trovato
si sono utilizzati dei test statistici, di cui l'articolo\cite{10.1145/1321440.1321528}
ha consigliato.

Per concludere si sono verificate alcune ipotesi
che erano state fatte dai tutor come premessa
del tirocinio stesso.

\subsection{Linguaggi utilizzati}
I linguaggi che sono stati utilizzati sono stati prevalentemente Java e Python.
Java è stato utilizzato per estendere le funzionalità della piattaforma Terrier,
mentre Python per eseguire alcuni esperimenti statistici.
Non sono mancate anche le competenze di scrittura di script
bash e di amministrazione di sistemi Linux, sui quali sono stati
eseguiti gli esperimenti prodotti.

\subsection{Ambienti di sviluppo}
Per sviluppare l'estensione di Terrier sono stati utilizzati principalmente
l'Ambiente di Sviluppo Integrato \textit{IntelliJ} e \textit{Maven},
come gestore dei pacchetti di Java, che
è lo standard adottato dalla piattaforma.
Per la ricerca e la sperimentazione è stato fatto uso di Jupyter Notebook
e Matlab, che si sono rivelati utili per la presentazione dei grafici
e dei test statistici.
