\chapter{Terrier}
Terrier (Terabyte Retriever) è una piattaforma di IR sviluppata dall'università di Glasgow, per permettere ai ricercatori di sperimentare su
grosse quantità di dati ed avere anche un modo per poter testare il sistema in tempo reale.
Tale piattaforma è scritta interamente in Java ed è open source, il cui codice sorgente è disponibile
interamente su \href{https://github.com/terrier-org/terrier-core}{GitHub}.
\\
Terrier è stato utilizzato come mezzo di sperimentazione, infatti
una parte del tirocinio è stata quella di estenderlo, per poter eseguire un processo
di \textit{model selection}.

La sua struttura modulare consente agli sviluppatori di aggiungere qualsiasi tipo di funzionalità con una
più evoluta.
Principalmente la piattaforma riprende l'architettura base di un sistema di IR, ovvero la suddivisione in
\textit{Indexing} e \textit{Retrieving}. La parte di \textit{Indexing} non è stata modificata, poichè
BM25P richiede un tipo di indexer a blocchi, il quale è già implementato nella libreria di sistema.
La parte che invece è stata estesa è la parte di \textit{Retrieving} e \textit{Evaluation}. 
\'E stato cioè implementato un meccanismo per fare model selection utilizzando svariati algoritmi, di cui
ne parleremo in seguito. 

\section{Struttura della piattaforma}
Al fine di garantire all'utente più modi per lavorare con Terrier, gli sviluppatori
permettono di scaricare una versione precompilata e una versione da compilare con
Maven. Entrambe le versioni però suggeriscono la costruzione
di una cartella di progetto con la seguente struttura:

\pagebreak

\begin{table}[h]
	\centering
	\begin{tabular}{|c|c|}
		\hline
		\textbf{Nome directory} & \textbf{Descrizione} \\
		\hline
		bin & contiene tutti i file binari per l'avvio \\
		\hline
		src & contiene i file sorgente \\
		\hline
		etc & destinata a contenere i file di configurazione \\
		\hline
		var & destinata ai file generati in output \\
		\hline
		share & contiene i file del dataset \\
		\hline
	\end{tabular}
\end{table}

Se si decide di compilare Terrier in autonomia è possibile farlo, poichè con il comando \textit{mvn package},
verrà costruito il file \textit{.jar} al quale gli script di avvio nella cartella \textit{bin} fanno riferimento.

\paragraph{Configurazione}
Tutta la configurazione del motore di ricerca è collocata nel file \textit{etc/terrier.properties}, il quale
è in formato \textit{Properties} di Java.

\section{Pipeline}
\subparagraph{Indicizzazione} Affinchè il motore di ricerca sia pronto per eseguire una ricerca,
Terrier deve costruire l'indice, partendo
da una collezione di documenti. Essa è rappresentata con un file con l'estensione \textit{trec},
il quale acronimo sta per Text Retrieval Conference.
La sintassi è simile a quella di XML e il file è strutturato nel seguente modo:

\begin{lstlisting}
<DOC>
<DOCNUM>docNum</DOCNUM>
Corpo del documento
</DOC>
...
<DOC>
...
</DOC>
\end{lstlisting}

Trec è un formato standard, pertanto tutti gli enti incaricati
nella loro realizzazione distribuiscono i dataset anche in questo
modo, garantendo a tutti gli utenti la compatibilità tra motori
di ricerca sperimentali.

Il comando per avviare l'indicizzazione è:

\begin{lstlisting}{bash}
user@terrier: bin/terrier batchindexing
\end{lstlisting}

\pagebreak

\subparagraph{Retrieving}
La fase di retrieve è la fase in cui Terrier, in modalità batch, esegue le query
e scrive i risultati su un file di output.
Le query sono tutte situate in un file chiamato \textit{topic file}, il quale
è in un formato speciale, così composto:
\begin{lstlisting}
<queryID_1>:<token_1> ... <token_k>
<queryID_2>:<token_1> ... <token_k>
...
<queryID_n>:<token_1> ... <token_k>
\end{lstlisting}

L'output prodotto dai risultati della query, viene
nel file \textit{results.txt} nella cartella \textit{var/output}. Il formato è testuale
ed ogni riga contiene una quadrupla di informazioni quali
$\langle idQuery, idDocumento, indiceRisultato, rank \rangle$.
Da notare che solitamente Terrier richiede un limite superiore
al numero di documenti da mettere in lista, ed esso è identificato
con $r_n$. Per essere più chiari di seguente riportiamo un esempio
di output file.

Il comando per eseguire la fase di retrieve è il seguente:
\begin{lstlisting}{bash}
user@terrier: bin/terrier batchretrieve
\end{lstlisting}

\begin{esempio}
	Assumiamo che $r_n=3$, pertanto l'output sarà della forma:
	
	\begin{lstlisting}
	<query_0> <doc_ID> 0 <rank>
	<query_0> <doc_ID> 1 <rank>
	<query_0> <doc_ID> 2 <rank>
	<query_1> <doc_ID> 0 <rank>
	...
	<query_1> <doc_ID> 2 <rank>
	...
	<query_n> <doc_ID> 3 <rank>
	\end{lstlisting}
\end{esempio}

\subparagraph{Valutazione} La fase di valutazione, che chiameremo in gergo tecnico evaluation,
è l'ultima della pipeline di Terrier, la quale si basa sul valutare la qualità dei risultati.
Ogni dataset per information retrieval contiene un file chiamato \textit{qrels}, il cui
acronimo è quello di query relevance, cioè ``rilevanze delle query".
Come illustrato nel capito precedente, per valutare un motore di ricerca bisogna
avere sia i risultati del modello, sia le rilevanze stabilite dagli esseri umani.
Questa fase impiega dunque due sorgenti di file, i risultati e le rilevanze e produce
la valutazione richiesta (per esempio recall, NDCG o altre).
Terrier nelle ultime versioni ha implementato anche tale fase,
che inizialmente era un applicativo a parte chiamato \textit{trec\_eval}.\footnote{\'E disponibile il sorgente in liguaggio C su Github all'indirizzo \href{https://github.com/usnistgov/trec_eval}{https://github.com/usnistgov/trec\_eval}}

Il file qrels è un file con un formato uguale a quello dei risultati,
con l'unica differenza che esso contiene la rilevanza espressa dall'utente invece che il rank.

Per valutazione il comando è il seguente
\begin{lstlisting}{bash}
user@terrier: bin/terrier trec_eval <qrels_filepath> <results_filepath> -m <measures>
\end{lstlisting}

dove mesaures indica un elenco di misure.
